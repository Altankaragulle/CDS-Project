\documentclass[12pt,a4paper]{article}

% ============================================================================
% PACKAGES
% ============================================================================
\usepackage[utf8]{inputenc}
\usepackage[T1]{fontenc}
\usepackage[english]{babel}
\usepackage{amsmath,amssymb,amsthm}
\usepackage{graphicx}
\usepackage{float}
\usepackage{subcaption}
\usepackage{caption}
\usepackage{booktabs}
\usepackage{multirow}
\usepackage{array}
\usepackage{longtable}
\usepackage{xcolor}
\usepackage{hyperref}
\usepackage{geometry}
\usepackage{setspace}
\usepackage{enumitem}
\usepackage{fancyhdr}
\usepackage{titlesec}
\usepackage[numbers,sort&compress]{natbib}

% ============================================================================
% PAGE SETUP
% ============================================================================
\geometry{left=2.5cm, right=2.5cm, top=2.5cm, bottom=2.5cm}
\onehalfspacing

% Figure path
\graphicspath{{./report/figures/}{./report/figures/experiments/}}

% Caption formatting
\captionsetup{font=small, labelfont=bf, format=plain, justification=justified, singlelinecheck=false}

% Hyperref setup
\hypersetup{colorlinks=true, linkcolor=blue, citecolor=blue, urlcolor=blue, bookmarksnumbered=true, pdfstartview=FitH}

% Header and footer
\pagestyle{fancy}
\fancyhf{}
\fancyhead[L]{\leftmark}
\fancyhead[R]{\thepage}
\renewcommand{\headrulewidth}{0.4pt}

% Section formatting
\titleformat{\section}{\normalfont\Large\bfseries}{\thesection}{1em}{}
\titleformat{\subsection}{\normalfont\large\bfseries}{\thesubsection}{1em}{}

% ============================================================================
% TITLE PAGE
% ============================================================================
\title{
    \textbf{\Large Predicting Corporate Distress with Machine Learning:} \\
    \vspace{0.3cm}
    \textbf{\Large Beyond Naive CDS Heuristics} \\
    \vspace{1cm}
    \large Master's Thesis \\
    Corporate Data Science -- Advanced Programming
}

\author{
    Altan Karagulle \\
    \small Supervisor: TA Francesco \\
    \small [University Name] \\
    \small \today
}

\date{}

% ============================================================================
% DOCUMENT
% ============================================================================
\begin{document}

\maketitle
\thispagestyle{empty}

\newpage
\tableofcontents
\listoffigures
\listoftables
\newpage

% ============================================================================
% ABSTRACT
% ============================================================================
\begin{abstract}
\noindent
This thesis investigates whether machine learning models can predict corporate distress more accurately than simple heuristics based on credit default swap (CDS) spreads. We manually curate a dataset of approximately 300 US firms (2010--2023) by integrating Compustat fundamentals, CRSP market data, and Markit CDS spreads through a systematic 15-step pipeline. We define distress as a CDS spread widening of at least 50 basis points or 25\% over a 12-month horizon, aligning with Basel III risk management practices.

Our incremental value analysis reveals that machine learning adds substantial predictive power: a naive ``high CDS = high risk'' rule achieves 0.471 AUC, while ML trained on CDS features alone improves to 0.550 AUC (+16.8\%). Incorporating accounting and market fundamentals further increases performance to 0.568 AUC (+20.6\% total improvement). The final optimized model achieves 0.658 AUC on the held-out test set (2021--2023), representing a 39.7\% improvement over the naive baseline.

We validate robustness through 5-fold time-series cross-validation, demonstrating mean AUC of 0.542 with regime-dependent performance. Probability calibration via isotonic regression reduces expected calibration error to 0.014, making predictions suitable for real-world capital allocation. Our results demonstrate that ML extracts significantly more predictive signal from credit data than practitioners' simple rules, with business impact including 47\% fewer false alarms and more efficient resource allocation.

\vspace{0.5cm}
\noindent \textbf{Keywords:} Corporate distress prediction, Credit default swaps, Machine learning, LightGBM, Probability calibration, Incremental value analysis
\end{abstract}

\newpage

% ============================================================================
% Include section files
% ============================================================================
\input{sections/01_introduction}
\input{sections/02_literature}
\input{sections/03_data}
\input{sections/04_variables}
\input{sections/05_methodology}
\input{sections/06_experiments}
\input{sections/07_calibration}
\input{sections/08_evaluation}
\input{sections/09_limitations}
\input{sections/10_conclusion}

% ============================================================================
% BIBLIOGRAPHY
% ============================================================================
\bibliographystyle{plainnat}
\bibliography{references}

\end{document}
